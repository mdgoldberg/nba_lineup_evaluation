%!TEX root = ../thesis.tex
% the abstract

NBA coaches and general managers are tasked with building lineups and rosters that
maximize their chances of winning. Further, basketball is a team sport where
interactions between the players in a lineup can be integral to the success or
failure of that lineup; managing a team is not as simple as building a lineup of the
best possible players without considering the way in which their play styles
interact with one another. However, the most prominent statistical methods for
evaluating players make the explicit assumption that there are no interaction
effects between the players on the court in producing point outcomes. This thesis
introduces a model to predict the expected point differential between two
five-player lineups in the NBA based on the players' play styles and the way in
which these play styles interact, where play styles are represented by a collection
of statistics collected from play-by-play data to describe each player's tendencies.
This model, which is shown to give a significant predictive improvement over today's
standard adjusted plus/minus models for player evaluation, is then used to evaluate
players, lineups, and teams.
