%!TEX root = ../thesis.tex
\chapter{Conclusion}
\label{conclusion}

\newthought{As shown in chapter~\ref{ch:results}, the model} trained on player
profile data provided significantly better out-of-sample prediction accuracy
compared to the baseline RAPM model. This provides evidence for the hypothesis that
chemistry and complementary play styles are critical in order to be successful in
the NBA.  Indeed, while RAPM certainly has its use cases, such as when attempting to
break down team outcomes and distribute responsibility to the individual level, the
success of the model herein suggests that there is more to building a strong lineup
than simply maximizing the squad's combined RAPM ratings.

The primary implication of this work is that it indicates that NBA coaches and
executives should dedicate more resources to understanding the way in which players'
play styles and abilities interact with one another from a statistical perspective.
Often, the focus in the NBA is solely on acquiring star talent; instead, teams
should make it a first-class goal to build a roster that allows the coach to play
players with complementary play styles. As seen in section~\ref{sec:lineup_ratings},
some lineups are able to overachieve their RAPM expectations due to the way that
their play styles interact; this fact could be advantageous for a team who wishes to
stay competitive while maintaining enough available cap space to sign high-impact
players in a coming offseason for example.  The principle of getting as much out of
one's players as possible is especially crucial when a coach has to turn to his
bench during a game. With twelve players available to play, there are 792 possible
lineups a coach can choose from during a game; choosing the optimal lineup based on
his players' play styles and on the lineup selected by the opposing team's coach
without taking a principled, analytical approach would thus be nearly impossible.
The results presented here motivate an increase in the degree to which basketball
analytics are used to make in-game decisions. We have seen similar movements in
other sports; for example, Microsoft has partnered with the National Football
League, where players and coaches use Microsft Surface tablets to employ analytics
software in real-time \cite{Geekwire}.  A similar movement in professional
basketball could allow coaches to utilize models such as the one presented here in
order to evaluate substitution decisions.

There is also room for extensions of the research presented here. One possible
extension would be to add additional features to the regression model. For example,
one drawback of the model is that it assumes that the point outcome is only a
function of the players on the court; additional features could include terms
measuring fatigue of a player, such as the number of possessions that he has been on
the court. The model could also incorporate player-tracking data, which the NBA
began collecting at the beginning of the 2013-14 NBA season; this data could be
incorporated into the player profiles and into the regression model directly. In
particular, the player-tracking data could increase the signal in the player
profiles; for example, one could quantify how a player moves without the ball or how
often a player is involved in different types of plays, such as pick-and-rolls.

Another extension of the research would be to analyze complementary play styles
within rosters. While the research here focuses on evaluating lineups, NBA
executives need to make decisions on a season level about which players should be
acquired and which should not; these decisions come in many forms, as players can be
acquired through trades, free agency, or the NBA draft. In constructing a roster,
the goal is to acquire talented players while giving the coach more flexibility in
his lineup and substitution decisions; therefore, the model presented in this thesis
could be extended to determine the cohesion of a roster's play styles. Similarly,
this model could be used to analyze whether a trade would have a positive impact on
a team by determining how the new player or players would complement the players
that remain on the roster after the trade.

While the conclusion that interactions between players' skill sets make a
significant impact on the outcome of possessions and games in the NBA is not
remarkable in itself, it is an important fact to consider when thinking about or
watching basketball. Indeed, the results of this research confirm systematically
what many NBA fans, analysts, and teams have assumed intuitively since the game was
invented: there is more to a lineup's value than the simple sum of its parts.
