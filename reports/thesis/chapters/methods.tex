%!TEX root = ../thesis.tex

\chapter{Methods}
\label{ch:methods}

\newthought{At a high level, the goal of this analysis} is to determine how
effective a lineup will be, taking into account the talents and tendencies of all
ten players on the court. This involves two main steps: first, there must be a
quantitative representation of each player's abilities and play style. Then, these
``player profiles'' of the players on the court can be used to predict the outcome
of each possession, measured in terms of points scored. Therefore, the problem can
be stated as a regression with each possession as a unit of observation, points
scored as the response variable, and features describing the play styles of the
players on the court during the possession as predictors.

In order to quantify play styles for prediction within season $t$, player profiles
were constructed from play-by-play-based statistics from both season $t-1$ and the
first half of season $t$; this process is described in section \ref{sec:profiles}.
Then, these play styles were passed through a dimensionality reduction algorithm in
order to find latent features that describe a player's play style more succinctly
and with less multicollinearity; this process is described in section
\ref{sec:dim_red}. Finally, these dimensionality-reduced features were used to
predict point outcomes for each possession in the second half of season $t$; this
process is described in \ref{sec:regress}.

\section{Building Player Profiles Representative of Play Style}
\label{sec:profiles}

* general idea: use first half play to predict second half performance
    - advantage: more current data, allows rookies to be included
* also: we need a "replacement player" as a baseline for comparison
* issue: injuries could result in a player being treated as replacement level
because he didn't play in the first half
* solution: use weighted blend of previous year and current year
    - this incorporates more data and enables us to include injured stars
* for every statistic, compute last year + weight * first half this year
    - effectively gives weight/2 weight to this year and 1 weight to last year
* tech fouls and tech FTA were ignored
* break features into offense, defense, rebounding, and RAPM
    - go section by section

\section{Finding Latent Features via Dimensionality Reduction}
\label{sec:dim_red}
This is a test

\section{Predicting Point Differential Based on Lineup Composition}
\label{sec:regress}
This is a test
