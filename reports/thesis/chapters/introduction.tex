%!TEX root = ../thesis.tex
\chapter{Introduction}
\label{introduction}

Measuring a individual player's contribution to his team's success is a central
question in basketball analytics. National Basketball Association (NBA) coaches
and executives are tasked with building rosters and lineups that maximize their
chances of winning a championship, so understanding which players make the
biggest positive contributions towards that goal is critical. In a team game
such as basketball, however, it can be difficult to attribute team success to
the individual players on a roster; indeed, confounding can be a big problem
when five players play at a time, and up to twelve players can play for a team
in any given game. A common solution to this issue is to look at per-game box
score statistics such as points per game, rebounds per game, and assists per
game. While this is a serviceable starting point to determine the league's
superstars from its benchwarmers, these summary statistics are unable to
capture most of the events that happen on the court; for example, they are
unable to measure many of a player's contributions on the defensive end, and
they reveal little about a player's role in offensive possessions, other than as
a scorer or an assister. Indeed, a player's contributions made through off-ball
defense, setting picks, and spacing the floor are completley uncaptured by
basic box score statistics.

In an attempt to more completely summarize the impact of a player, basketball
analysts turned to the concept of plus/minus. Plus/minus statistics begin from
the fact that basketball is ultimately a game of points: the goal is to outscore
one's opponent, or in other words, to achieve a positive point differential.
The plus/minus framework extends this fact to the player level; fundamentally,
plus/minus attempts to measure the point differential attributable to a given
player. The most basic form of plus/minus is raw plus/minus, which is simply the
difference between a player's team's points and his opponent's points while the
player is on the floor. Raw plus/minus was first used in hockey in the mid-20th
century and became an official National Hockey League statistic in 1968; many
years later, the statistic was introduced in basketball and was ultimately
adopted by the NBA in 2003.

Although this metric has the correct intention, it has several flaws. First, it
fails to consider the relative strength of the team on which a player plays. For
example, Anthony Davis is a superstar for the New Orleans Pelicans but had a raw
plus/minus of -200 in the 2015-16 season. It would be misguided to conclude that
Anthony Davis is a negative influence on the Pelicans; in fact, all but one
player on the Pelicans had a negative raw plus/minus. A second flaw of raw
plus/minus is that it fails to consider the teammates with which and the
opponents against which a player plays. This can be an issue when evaluating
lineups containing particularly good or bad players; for example, Udonis Haslem
was able to accumulate a raw plus/minus of +272 with the 2012-13 Miami Heat, on
which Haslem frequently played with stars like Lebron James, Dwyane Wade, and
Chris Bosh. Moreover, raw plus/minus fails to consider the quality of one's
opponents, so a player who frequently plays against another team's starters will
be judged the same as a player who typically plays against another team's
backups.

Many extensions and modifications of raw plus/minus have been made to overcome
these flaws. The first such metric is net plus/minus, which is defined for a
player as a team's point differential when the player is on the court minus the
team's point differential when the player is not on the court. This alleviates
the first concern about raw plus/minus by comparing the player's raw plus/minus
to a baseline of the team's point differential without the player, rather than
raw plus/minus's implicit baseline of 0 (note that when a team's point
differential without a player is zero, that player's net plus/minus is
equivalent to his raw plus/minus). Therefore, when a team is particularly bad,
their players who contribute relatively positively are still able to achieve a
positive plus/minus, as their team is better with them than without them.
Likewise, a poor player who achieves a high raw plus/minus by simply playing
with a great team that accumulates a high overall point differential is not
rewarded the same way under net plus/minus. Unfortunately, although net
plus/minus does address the first issue with raw plus/minus, it fails to account
for a player's teammate and opponent quality, so players who frequently play
with or against particularly great or mediocre players will be judged by net
plus/minus in part by with and against whom they play, rather than by their own
individual contributions.

To explicitly control for a player's teammates and opponents, analysts have
extended the plus/minus framework even further to create a metric called
adjusted plus/minus (APM). APM accounts for the shortcomings of raw and net
plus/minus by explicitly controlling for a player's teammates and opponents on
each possession in order to estimate the player's effect on the team's point
differential, compared to an average player. APM is unique in that its
computation is not as straightforward as simply aggregating point differentials;
instead, it is computed using a least-squares linear regression where the unit
of observation is a possession, the predictors are indicators representing
whether a given player is on the floor, and the response variable is the point
differential for the possession. Fitting such a model results in coefficient
estimates for each player in the data set, where the sum of one lineup's
coefficients minus the sum of the opposing team's lineup's coefficients is the
model's estimate of the expected point differential between the two lineups
(typically prorated to 100 possessions). By including predictors for every
player on the court, it is able to "adjust" each player's plus/minus for the
quality of the players he plays with which and against which he plays, thereby
statistically isolating the player's contribution. This is much more effective
in capturing the contributions a player can make that do not appear in the box
score's summary statistics.

While APM provides a great improvement over far-simpler metrics like raw and net
plus/minus, it does have some problems in its most basic form. One issue is
multicollinearity, or high correlation between the predictors; when certain
players are very frequently or very rarely on the court at the same time, it can
lead to numerical instability and thus can cause problems during estimation.
Another problem with APM is that it tends to overfit the data; when the number
of training observations (in this case, possessions) is relatively low compared
to the number of parameters being estimated (the number of players involved in
the sample), simple linear regression tends to fit the data too closely. This
results in a failure to generalize when making out-of-sample predictions, which
is critical if APM is to be trusted as a predictive model (i.e., indicative of
future performance). Finally, adjusted plus/minus assumes a player's level of
play is constant over time for the duration of time used in the training data.
This assumption is inherent to the model, and while in reality, players' skill
levels likely fluctuate in the short-term and certainly vary over the course
of a career, this tends to be a reasonable simplifying assumption for the
problem at hand.

TODO: RAPM (regularized APM - fixes collinearity and overfitting)

TODO: xRAPM (prior-informed method with different priors for each player, based
on box score statistics/previous years)

% maybe TODO: BPM (approximation of RAPM using advanced box-score metrics)

These plus/minus metrics measure an individual player's contribution with
varying degrees of success. However, we return the question that NBA all coaches
and general managers face: how does a team best maximize its chances of winning
a championship? As discussed previously, winning in basketball comes down to
outscoring one's opponent, so this question reduces to one of team point
differential. For a coach, this in essence boils down to deciding which five
players should take the court at any given time in order to maximize one's
chances of winning a game. For a general manager, it boils down to deciding how
to construct a roster that maximizes one's chance of winning. Before, we went on
to answer these question by examining each individual player's contribution to
the team's overall point differential.  However, if the reason we care about an
individual's value is because it is integral to team success, then we ought to
view the player in the context of his team. After all, basketball is a team
sport, and coaches and executives in the NBA do not make their decisions about
players in a vacuum; rather, they make decisions based on a player’s value in
the context of the team they already have. Indeed, chemistry is often critical
to success in the NBA, so there is a strong incentive to ensure that players on
a team fit well together. Therefore, it is crucial to analyze not only how much
an individual player contributes to team plus/minus, but also how these players'
interactions result in the team's overall point differential. I will first
review previous work analyzing lineups and interactions between players, and
then I will outline my approach for answering these questions.

TODO: brief review of Maymin et al (2012): "NBA Chemistry: Positive and Negative
Synergies in Basketball"

TODO: brief review of Joseph Kuehn (2016): "Accounting for Complementary Skill
Sets When Evaluating NBA Players' Values to a Specific Team"

TODO: brief review of Arcidiacono et al (2015): "Productivity Spillovers in Team
Production: Evidence from Professional Basketball"
