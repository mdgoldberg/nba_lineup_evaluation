\title{CS91r Final Submission}
\author{Matt Goldberg}
\date{\today}

\documentclass[12pt]{article}

\usepackage[margin=1.0in]{geometry}
\usepackage{setspace}
\usepackage{enumerate}
\usepackage{amsmath, amssymb, amsthm, mathtools}

% Statistics
\newcommand{\Prob}[1]{P\left( #1 \right)}
\newcommand{\PDF}[1]{f\left( #1 \right)}
\newcommand{\E}[1]{\mathbb{E}\left[ #1 \right]}
\newcommand{\var}[1]{\mathrm{Var}\left[ #1 \right]}
\newcommand{\cov}[2]{\mathrm{Cov}\left[ #1 , #2 \right]}
\newcommand{\iid}{\textrm{ i.i.d.}}
\newcommand{\simiid}{\overset\iid\sim}
% Distributions
\newcommand{\Norm}[2]{\mathrm{\mathcal{N}\left(#1, #2\right)}}
\newcommand{\Bin}[2]{\mathrm{Bin\left(#1, #2\right)}}
\newcommand{\Bern}[1]{\mathrm{Bern\left( #1 \right)}}
\newcommand{\Pois}[1]{\mathrm{Pois\left( #1 \right)}}
\newcommand{\Expo}[1]{\mathrm{Expo\left( #1 \right)}}
\newcommand{\GammaDist}[2]{\mathrm{Gamma\left( #1, #2 \right)}}
% Sets
\newcommand{\R}{\mathbb{Z}} % reals
\newcommand{\Z}{\mathbb{Z}} % integers
\newcommand{\N}{\mathbb{N}} % natural numbers
\newcommand{\F}[1]{\mathcal{F}_{#1}}
% Other Math
\DeclareMathOperator*{\argmin}{arg\,min}
\DeclareMathOperator*{\argmax}{arg\,max}
\newcommand{\myexp}[1]{\exp\left[ #1 \right]}
\newcommand{\FirstDeriv}[1]{\frac{\partial}{\partial #1}}
\newcommand{\SecondDeriv}[1]{\frac{\partial^2}{\partial (#1)^2}}
% if/otherwise for cases environment
\newcommand{\myif}{& \textrm{if }}
\newcommand{\myotherwise}{& \textrm{otherwise}}


\begin{document}
\maketitle

\doublespacing

\section{Introduction} \label{intro}

Much of the work being done in the world of NBA analytics attempts to answer
the problem of how an individual player should be valued. In particular, a
common insight is that basketball is inherently a team sport, and the players
with whom and against whom one plays can significantly alter one's traditional
box score statistics. Therefore, a major problem in player evaluation is
teasing apart an individual player's contribution to his team without falsely
attributing contributions to poor players who merely play with great players
(or against even worse players), and without overlooking great players on
mediocre teams. The major breakthrough in player evaluation that addressed
these issues was Adjusted Plus/Minus\cite{rosenbaum_2004}, which controls for a
player's teammates and opponents to determine what effect a player has on point
differential, independent of who else is on the court. Since this framework was
introduced, many others have expanded on it and adopted the mindset of breaking
down plus/minus \cite{englemann_2015}\cite{sill_2010}\cite{myers_2011}.

However, NBA general managers and coaches do not make decisions about a player
based on his value in a vacuum; rather, they make decisions based on a player's
value in the context of the team they already have. Notably, chemistry is often
critical to success in the NBA, so there is a strong incentive to ensure that
players fit well together\cite{schrage_2014}. Therefore, my goal is to build a
principled model that, at its core, predicts the point differential between a
given two lineups, based on the talent and play style of the players involved.
Using this model, a coach would be able to determine how player's specific
skillset would make an impact given their current roster. Moreover, a coach
could determine which players on a roster would complement each other well and
should play a significant portion of their minutes together.

\section{Background} \label{background}

\begin{itemize}
	\item Explain Adjusted Plus/Minus and RAPM
	\item Clustering is very common - "new positions"
	\item Explain and compare/contrast \cite{maymin_2012}
	\item Explain idea of regression with RAPM and play styles
\end{itemize}

\section{Clustering Approach} \label{clustering}

\subsection{Feature Selection} \label{features}

* Explain that we want style as much as possible, without taking into account
minutes, actual contribution, etc.
* Go over different areas of play style and stats to represent each

\subsection{Clustering Algorithms} \label{algos}

* Go over different clustering algorithms briefly
* Use of silhouette score for evaluation \cite{rousseeuw_1987}

KMeans

GMM

PCA + KMeans

PCA + GMM

\section{Results} \label{results}

* See notebook
* Visualizations (?)

\section{Future Work} \label{future}

* Regression
\begin{itemize}
	\item Given such a system, possible insights and analysis: Section \ref{future}
		\subitem What is each team's best lineup? Especially relevant when starters
		are resting
		\subitem Are certain types of players more valuable, or perhaps more
		replaceable, than others?
		\subitem Which coaches are the best at playing players that complement each
		other?
		\subitem Given a roster and a possible acquisition, how valuable is the
		player in question? How well do they fit?
		\subitem Are there any notable mutually-beneficial trades that could happen?
		\subitem How have these effects changed over time?
\end{itemize}

\bibliographystyle{plain}
\bibliography{sources}

\end{document}
