\title{CS91r Final Submission}
\author{Matt Goldberg}
\date{\today}

\documentclass[12pt]{article}

\usepackage[margin=1.0in]{geometry}
\usepackage{setspace}
\usepackage[parfill]{parskip}
\usepackage{enumerate}
\usepackage{amsmath, amssymb, amsthm, mathtools}

% Statistics
\newcommand{\Prob}[1]{P\left( #1 \right)}
\newcommand{\PDF}[1]{f\left( #1 \right)}
\newcommand{\E}[1]{\mathbb{E}\left[ #1 \right]}
\newcommand{\var}[1]{\mathrm{Var}\left[ #1 \right]}
\newcommand{\cov}[2]{\mathrm{Cov}\left[ #1 , #2 \right]}
\newcommand{\iid}{\textrm{ i.i.d.}}
\newcommand{\simiid}{\overset\iid\sim}
% Distributions
\newcommand{\Norm}[2]{\mathrm{\mathcal{N}\left(#1, #2\right)}}
\newcommand{\Bin}[2]{\mathrm{Bin\left(#1, #2\right)}}
\newcommand{\Bern}[1]{\mathrm{Bern\left( #1 \right)}}
\newcommand{\Pois}[1]{\mathrm{Pois\left( #1 \right)}}
\newcommand{\Expo}[1]{\mathrm{Expo\left( #1 \right)}}
\newcommand{\GammaDist}[2]{\mathrm{Gamma\left( #1, #2 \right)}}
% Sets
\newcommand{\R}{\mathbb{Z}} % reals
\newcommand{\Z}{\mathbb{Z}} % integers
\newcommand{\N}{\mathbb{N}} % natural numbers
\newcommand{\F}[1]{\mathcal{F}_{#1}}
% Other Math
\DeclareMathOperator*{\argmin}{arg\,min}
\DeclareMathOperator*{\argmax}{arg\,max}
\newcommand{\myexp}[1]{\exp\left[ #1 \right]}
\newcommand{\FirstDeriv}[1]{\frac{\partial}{\partial #1}}
\newcommand{\SecondDeriv}[1]{\frac{\partial^2}{\partial (#1)^2}}
% if/otherwise for cases environment
\newcommand{\myif}{& \textrm{if }}
\newcommand{\myotherwise}{& \textrm{otherwise}}


\begin{document}
\maketitle

\doublespacing

\section{Introduction} \label{intro}

Much of the work being done in the world of NBA analytics attempts to answer
the problem of how an individual player should be valued. In particular, a
common insight is that basketball is inherently a team sport, and the players
with whom and against whom one plays can significantly alter one's traditional
box score statistics. Therefore, a major problem in player evaluation is
teasing apart an individual player's contribution to his team without falsely
attributing contributions to poor players who merely play with great players
(or against even worse players), and without overlooking great players on
mediocre teams. The major breakthrough in player evaluation that addressed
these issues was adjusted plus/minus\cite{rosenbaum_2004}, which controls for a
player's teammates and opponents to determine what effect a player has on point
differential, independent of who else is on the court. Since this framework was
introduced, many others have expanded on it and adopted the mindset of breaking
down plus/minus \cite{englemann_2015}\cite{sill_2010}\cite{myers_2011}.

However, NBA general managers and coaches do not make decisions about a player
based on his value in a vacuum; rather, they make decisions based on a player's
value in the context of the team they already have. Notably, chemistry is often
critical to success in the NBA, so there is a strong incentive to ensure that
players fit well together\cite{schrage_2014}. Therefore, my goal is to build a
principled model that, at its core, predicts the point differential between a
given two lineups, based on the talent and play style of the players involved.
Using this model, a coach would be able to determine how player's specific
skillset would make an impact given their current roster. Moreover, a coach
could determine which players on a roster would complement each other well and
should play a significant portion of their minutes together.

\section{Background} \label{background}

One related piece of literature is a paper presented at Sloan Sports Analytics
Conference in 2012 by Allan Maymin, Philip Maymin, and Eugene
Shen\cite{maymin_2012}. In this paper, the authors used a skills plus/minus
framework to evaluate a player's offensive and defensive capabilities in
scoring, rebounding, and ball-handling. Then, they used these skillset metrics
to simulate games, which measured the success of the individual players
involved as well as the overall lineup. While the goal of this paper is very
similar to mine, they took a different approach than the one that I plan on
taking; most notably, I am using clustering rather than regression to
approximate play styles and I am using regression rather than simulation to
evaluate players and lineups. Nonetheless, this paper is very inspiring in
terms of the questions it sets out to answer, so it will surely aid me
throughout my thesis. I will now lay out the basic setup of my model.

First, adjusted plus/minus is essentially a regression on play-by-play data
where each possession is an observation, and the predictors are indicators for
each player on the court on the home team and each player on the court for the
away team. Moreover, regularized adjusted plus/minus expands on adjusted
plus/minus by utilizing ridge regression instead of ordinary least squares,
thus preventing overfitting for players with low sample sizes and adding
stability to the problem. I will be using some form of APM or RAPM as a proxy
for talent, or more precisely, for how much a player contributes to his team,
regardless of his play style and independent of his teammates on the court.

Second, I will use various clustering techniques to ascertain each player's
play style. More specifically, I will select features that are indicative of a
player's skillset and how they play the game. This is a very common technique
in basketball analytics, and it is frequently used to cluster players into
``positions'' outside the traditional five positions\cite{lutz_2012}. Moreover,
I will apply a clustering algorithm such that each player-season is assigned to
a cluster, allowing a player's play-style to change each season. This will give
a proxy for what kind of player someone is, without speaking to how talented
the individual is.

Finally, I will combine these two measures of each player into an APM-like
regression in which the dependent variable is point differential and the
predictors are proxies for each player's talent (i.e. APM) and play style
(i.e., output from the clustering algorithm). More specifically, I plan to
predict point differential in a game in year $Y$ using play style and APM data
from year $Y-1$. Using this data for the players on the court, I will then
predict point differential in order to ascertain the impact of different play
styles and different levels of skill on point differential.


\section{Clustering Approach} \label{clustering}

This section will discuss the approach to clustering data in order to deduce
each player-season's play style.

\subsection{Feature Selection} \label{features}

In selecting player features on which to cluster, it is important to select
features that are as informative about a player's play style as possible. For
example, while points per game might be an important statistic elsewhere, it is
not particularly telling about a player's play style because it is highly
dependent on how many minutes he gets from his coach. Likewise, true shooting
percentage is not as informative about a player's play style than his
distribution of shot distances, since the latter says more about the player's
shot selection. Moreover, true shooting percentage and statistics like it are
likely to be correlated with APM, which is already included in the final
regression.

Keeping this in mind, I have currently selected 27 features to represent a
player's play style (at least, before possible dimensionality reduction). These features are summarized in Table \ref{tab:a}.

\begin{table}[!htb]
	\centering
	\begin{tabular}{|c|c|} \hline
		Area of Play & Features \\ \hline
		Vitals & Height, Weight \\ \hline
		FGA Distribution &
		\begin{minipage}[t]{0.6\columnwidth}
			Avg Shot Distance, Pct of FGA from 0-3 feet, 3-10, 10-16, 16+ foot
			2-pointers, and 3-pointers
		\end{minipage} \\ \hline
		FGM Distribution & FG\% from 0-3 feet, 3-10, 10-16, 16+ foot 2-pointers, and
		3-pointers \\ \hline
		Scoring Type & USG\%, free throw rate, percent of 2PM assisted,
		percent of 3PM assisted \\ \hline
		Rebounding & ORB\% DRB\% \\ \hline
		Defense & Defensive rating, BLK\%, STL\%, PF/poss \\ \hline
		Ball-Handling & AST\%, TO\%, bad passes per minute, lost balls per minute
		\\ \hline
	\end{tabular}
	\caption{Features Selected for Clustering} \label{tab:a}
\end{table}

There is a concern that the number of variables selected in each area of play
may implicitly give different weights to each area of play, depending on the
clustering method. For example, it may be the case that rebounding is
overlooked due to a relatively smaller number of features representing
rebounding. This is something I will keep my eye on, and it is something that
dimensionality reduction may be able to alleviate.

\subsection{Clustering Algorithms} \label{algos}

So far, I tried two different clustering algorithms: first, I tried standard
K-means. K-means is a clustering algorithm that iteratively assigns points to
the nearest cluster center and reassigns cluster centers to the centroid of the
points assigned to that cluster; the process ends when no points are assigned
to different clusters. Therefore, K-means maintains the property that any point
is part of the cluster with the closest center. Because K-means requires
computing distances between points, it is important or each dimension to have
the same units; therefore, I normalized the data before clustering by
subtracting the mean and dividing by the standard deviation of each feature.
This is also useful because it makes it easier to tell which players are above
or below average (and to what extent), which is especially useful for less
well-known features.

Second, I tried a Gaussian mixture model, a generative, probabilistic model. A
GMM is similar to K-means, except it models each component as a multivariate
Gaussian, meaning that different components can have different covariance
matrices, and thus different shapes. Moreover, it uses soft assignments in that
it models each point as a mixture of all $K$ components; in other words, it
allows a point to belong 20\% to one cluster and 80\% to another cluster. This
may be useful because it shows to what degree a player is between two clusters,
or a ``hybrid'' player.

I also tried to reduce the dimensionality of the feature space by applying PCA before applying clustering methods; I took the first 10 principal components, which explained about 80\% of the variance in the 27 features.

Finally, I evaluated how well a clustering fit the data by attempting to
maximizing the average silhouette score of the data, given a clustering\cite{rousseeuw_1987}.

\section{Results} \label{results}

For most of the results, please see the notebook that will be submitted with
this writeup, as it has both the analysis/clustering code and the results.
Empirically, based on the features I selected, there seems to be around 7
clusters in the data; it seems that fewer clusters actually fit the data
better, but in order to represent play styles effectively it is helpful to have
smaller, more granular clusters. This selection of the number of clusters will
be something that I figure out as I do the regression analysis, as the best
choice for $K$ will be the choice that allows the final regression to perform
best.

\section{Future Work} \label{future}

The next thing to work on for me is to finish up clustering experiments and
then to move on to formulating APM and creating a regression. Most importantly in this is scraping play-by-play data from basketball-reference.com effectively so that I can keep track of the lineup in the game at any point in time.

Other things I need to think about as I move forward with my thesis are the questions I want to answer once I have results. For example, possible questions I may wish to answer include:
\begin{itemize}
		\item What is each team's best lineup? This is especially helpful when
			starters are resting; what is the best lineup I can play without the
			players I need to rest? This could lead to a coaching decision model that
			incorporates fatigue as well as chemistry.
		\item Are certain types of players more valuable, or perhaps more
		replaceable, than others?
		\item Which coaches are the best at assembling rosters and playing players
			that complement each other well?
		\item Given a roster and a possible free agent/trade acquisition, how
			valuable is the deal in question? How well does it fit?
		\item Are there any notable mutually-beneficial trades that could happen?
		\item How have the effects of complementary skillsets changed over time?
\end{itemize}

Clearly, there are a lot of different ways I could take this project upon finishing the regression analysis.

\bibliographystyle{plain}
\bibliography{sources}

\end{document}
